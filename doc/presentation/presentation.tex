\documentclass[17pt,notes=hide,amsmath=namelimits,sumlimits,intlimits]{beamer}
\mode<presentation>

\usepackage[utf8]{inputenc}
\usepackage{lmodern}
\usepackage[T1]{fontenc}

\usetheme{Boadilla}
\useoutertheme{myinfolines}
\usecolortheme{rose}

\title[Tuchulcha]{Tuchulcha Project}
\author[J.M.Gottfried]{Jens-Malte Gottfried}
\date{19.02.2009}
\subject{Graphical Charon configuration tool}

\begin{document}

\begin{frame}
  \titlepage
\end{frame}

\begin{frame}
  \frametitle{Why do we need this?}
  \begin{itemize}
    \item algorithms are configured using plain text parameter files
    \item these files have to be created by hand
    \item a graphical tool could speed up and simplify
	  this configuration process
  \end{itemize}
\end{frame}

\begin{frame}
  \frametitle{Main focuses}
  \begin{itemize}
    \item written in C/C++ in order to integrate it
	  into the Charon project
    \item platform independent
    \item easy to learn
    \item show relevant class documentation
  \end{itemize}
\end{frame}

\begin{frame}
  \frametitle{Features}
  \begin{itemize}
    \item look and feel similar to other graphical IDEs
	  (LabView, QT Designer, ...)
    \item the graphical flowchart, which is easier to understand
	  than the plain config file, gives an overview of the data interaction
    \item documentation is always present on screen to help new users
    \item dock widgets can be rearranged or hidden to adapt to the user's needs
  \end{itemize}
\end{frame}

\begin{frame}
  \frametitle{Features}
  \begin{itemize}
    \item mouse navigation or keyboard shortcuts for advanced users
    \item the user's attention is focussed on the objects and parameter
	  configuration and not on syntax details and searching
	  the needed documentation
    \item the connectable item filter speeds up finding the needed object
  \end{itemize}
\end{frame}

\begin{frame}
  \frametitle{Features}
  \begin{itemize}
    \item multiple configuration can be opened at the same time and be
	  compared this way
    \item flowchart images can be saved and used e.g. for presentations
	  or documentation
    \item object/class information (so called metadata) are stored in a
	  separate file
	\item this way it's possible to configure visualizable
	  processes in many different projects/applications
  \end{itemize}
\end{frame}

\begin{frame}
  \frametitle{Used software}
  \begin{itemize}
    \item CMake build system
    \item Qt library (GUI, SVG, Webkit)
    \item Graphviz (flowchart visualization)
  \end{itemize}
\end{frame}

\begin{frame}
  \frametitle{Tested build environments}
  \begin{itemize}
    \item Linux gcc
    \item Win32 MinGW
    \item Visual Studio 2005
    \item Visual Studio 2008
  \end{itemize}
\end{frame}

\begin{frame}
  \frametitle{Future plans}
  \begin{itemize}
    \item possibility to hide or collapse parts of the
	  flow chart in order to focus on the relevant
	  parts which your are working on
    \item automatically metadata generation out of
	  existing class structures
    \item interaction with the Argos project to edit
	  and visualize the processing pipeline
  \end{itemize}
\end{frame}

\end{document}