\documentclass[a4paper, 11pt, fleqn, pointlessnumbers]{scrartcl}

\usepackage[utf8]{inputenc}   % zur richtigen Darstellung der Schriften
\usepackage[T1]{fontenc}      % T1-Font auswählen
\usepackage{moreverb}         % Quellcode einfügen
\usepackage{ngerman}          % Trennungsregeln
\usepackage{color}            % Hier kommt Farbe ins Spiel
\usepackage[german]{varioref} % flexible Querverweise
\usepackage{hyperref}         % Sprünge in der Datei

% links formatieren
\hypersetup{
  colorlinks,                 % Links farbig statt mit Rahmen
  linkcolor=black,            % z.B. im Inhaltsverzeichnis
  urlcolor=darkblue           % Weblinks
}

% Arbeitszeitangaben
\newcommand{\worktime}[1]{\hfill\textcolor{blue}{ca.~#1~h}}

\begin{document}
  \titlehead{
    Universität Heidelberg \hfill SoSem 2008\\
    Heidelberg Collaboratory for Image Processing (HCI)\\
    Speyerer Straße 4\\
    D-69115 Heidelberg
  }
  \subject{Projektpraktikum Informatik}
  \title{Charon Konfigurationseditor}
  \author{Jens-Malte Gottfried}
  \date{14.08.2008}
  \publishers{\vspace{1cm}Betreut durch Prof. Dr. B. Jähne}
  \maketitle

  \tableofcontents
  \pagebreak

  \section{Ziele}
  In diesem Projekt soll eine GUI-Anwendung erstellt werden, mit der
  man die umfangreichen Konfigurationsmöglichkeiten des Softwareprojektes
  "`Charon"' übersichtlich erzeugen, darstellen und bearbeiten kann.
  Hierbei ist Wert auf Plattformunabhängigkeit und Abstraktion zu legen,
  damit die entstehende Anwendung ohne größere Modifikationen auch für
  völlig andere Konfigurationsaufgaben verwendet werden kann. Ebenso soll
  die Anwendung leicht erweiterbar sein, um sie bei zukünftiger
  Weiterentwicklung des Charon Projektes anpassen zu können.

  \section{Einführung}
  Im Charon Projekt werden verschiedene Algorithmen, die u. A. zur
  Flussanalyse digitaler Bildsequenzen dienen, implementiert. Um die dort
  erstellten Anwendungen zu benutzen, müssen für die unterschiedlichen
  Algorithmen jeweils Konfigurationsdateien angelegt werden, in denen die zu
  verwendenden Parameter hinterlegt werden. Um z.B. eine Bildsequenz zu
  verarbeiten, werden die entsprechenden Daten geladen, mit einem Algorithmus
  bearbeitet, das Ergebnis an einen weiteren Algorithmus weitergegeben usw., bis
  schließlich das gewünschte Ergebnis berechnet wird. In diesem Projekt
  geht es nun darum, dieses Vorgehen zu visualisieren, d.h. den Datenfluss
  darzustellen, sowie die möglichen Algorithmen, die man auf die Daten
  anwenden kann, modular zusammenstellen zu können und die
  resultierenden Konfigurationsdateien zu erzeugen. Dabei soll die Anwendung
  sinnvolle Hinweise und Dokumentationen zu den einzelnen Algorithmen/
  Modulen anzeigen, die bei der Konfigurationserstellung weiterhelfen.

  \section{Programmbeschreibung}
  In diesem Projekt soll eine C++-Anwendung unter Verwendung des
  Qt-Framework der Firma Trolltech erstellt werden. Die Anwendung
  besteht aus einer graphischen Benutzeroberfläche, in deren Zentrum
  der zu konfigurierende Datenfluss ähnlich einem Datenflussdiagramm
  dargestellt wird. Wählt man ein Objekt im Flussdiagramm (d.h. ein zu
  konfigurierendes Modul im Datenfluss) aus, kann man dessen Eigenschaften in
  einem Objektinspektor betrachten und verändern. Der Objektinspektor
  speichert die Einstellungen in den dazugehörigen Konfigurationsdateien ab.
  Beim Erstellen des Datenflussschemas sollen sinnvolle Vorschläge, welche
  Module sich miteinander verknüpfen lassen, zur Verfügung stehen. Die nötigen
  Informationen, welche Module es überhaupt gibt und wie sie miteinander
  zusammenarbeiten können, werden in einer entsprechenden Datenbank zur
  Verfügung gestellt. Für jedes Modul und die zugehörigen Parameter
  sind Informationen zu hinterlegen, die bei der Konfiguration in einem
  Hilfe-Fenster angezeigt werden, um Anwendern die Benutzung zu
  erleichtern.

  Die Verwendung des Qt-Framework wurde ausgewählt, damit die
  Anwendung auf einer Vielzahl unterschiedlicher Plattformen lauffähig ist
  (angestrebt sind Win(32), Unix, Mac).

  Um die Software auf möglichst vielen Plattformen kompilieren zu können,
  wird das Build-System "`CMake"' verwendet, welches build-Konfigurationen
  für alle gängigen Entwicklungsumgebungen erzeugen kann. Hierzu gehören z.B.
  Unix/MinGW Makefiles oder Microsoft Visual C++ Projekte.

  Zur Dokumentation der Anwendung wird die Software "`Doxygen"' verwendet,
  die ebenfalls auf vielen Plattformen verfügbar ist und die Informationen
  aus speziellen Kommentaren innerhalb der Quelldateien generiert. Dies stellt
  sicher, dass die Dokumentation schon während des Schreibens des
  Programmcodes erstellt wird und nicht erst nachträglich hinzugefügt werden
  muss, was oft dazu führt, dass Inkonsistenzen zwischen Code und
  Dokumentation entstehen.

  Alle erwähnten Tools und Bibliotheken stehen als OpenSource unter GPL oder
  kompatiblen Lizenzen zur Verfügung, sodass keine Lizenzprobleme zu erwarten
  sind. Die entstehende Anwendung soll nach Abschluss des Projektes ebenfalls
  unter der GPL verfügbar gemacht werden, um die Weiterentwicklung zu ermöglichen.

  \section{Zeitplanung und Milestones}
  Die Anwendung besteht aus mehreren Modulen, die mehr oder weniger
  unabhängig voneinander entwickelt werden können. Der geschätzte
  Arbeitsaufwand ist rechts angegeben.
  \begin{itemize}
    \item Objektinspektor mit Anbindung an die Konfigurationsdateien
      \worktime{20}
    \item Datenflussdiagramm und -editor \worktime{20}\\
      dazu benötigt:
    \begin{itemize}
      \item Modulauswahlfenster \worktime{20}
      \item Visualisierung als Flussdiagramm \worktime{20}
    \end{itemize}
    \item Dokumentationsfenster (Anzeige der Hinweistexte) \worktime{20}
  \end{itemize}

  Zusätzlich zu den Programmkomponenten muss Folgendes erstellt werden:
  \begin{itemize}
    \item Datenbank der Informationen über die Datenflussstruktur
      \worktime{20}
    \item Dokumentationstexte zu den einzelnen Datenflussmodulen
      \worktime{20}
  \end{itemize}

  Weiterhin ist eine umfangreiche Programmdokumentation erforderlich,
  die die Weiterentwicklung und das Erweitern der Software ermöglicht und
  vereinfacht. \worktime{20}

  \vspace{5mm}
  Da in den Vorlesungen zwar Programmierkenntnisse vermittelt wurden,
  das umfangreiche Qt-Framework jedoch nicht behandelt wurde, gehört
  die Einarbeitung in Qt ebenfalls zur Bearbeitung dieses Projektpraktikums.
\end{document}
